\chapter{Introduction}

In the event of an outbreak it is important to have modelling tools in place to estimate the likely origin, speed of spread, and size, and to be able to predict the impact of intervention measures.  However, different diseases in different animal populations require somewhat different approaches due to the detection, transmission and recovery (or not) characteristics of hosts infected with the pathogen, and the contact patterns between susceptible hosts (whether of the same species or not).  Consequently specific models developed for one disease system are unlikely to be entirely suitable for another.  

Broadwick is a framework for developing sophisticated epidemiological based mathematical models, and consists of several Java libraries and bespoke packages.  The components of Broadwick are written in such a way that a scientist may combine them in order to rapidly prototype a model for a new specific scenario.

\begin{itemize}
\item Supports single (e.g. within herd) or structured populations (e.g. multi-species or locations)
\item Inclusion of movement over network data (e.g. Cattle movement Tracing System) 
\item Stochastic Individual Based simulations (including fast approximate options) 
\item Approximate Bayesian Computation inference for estimating model parameters from data via simulations
\item Markov Chain Monte Carlo inference for estimating model parameters from data
\end{itemize}

\section{License}\index{License}

Broadwick is released under the Apache 2 license.

%\section{Notations}\index{Notations}
%
%\begin{notation}
%Given an open subset $G$ of $\mathbb{R}^n$, the set of functions $\varphi$ are:
%\begin{enumerate}
%\item Bounded support $G$;
%\item Infinitely differentiable;
%\end{enumerate}
%a vector space is denoted by $\mathcal{D}(G)$. 
%\end{notation}

